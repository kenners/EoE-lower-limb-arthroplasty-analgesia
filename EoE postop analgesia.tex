\documentclass[11pt, a4paper]{article}

% LAYOUT
\usepackage{geometry}
\geometry{a4paper, margin=2.5cm}
\usepackage[parfill]{parskip}

\usepackage[UKenglish]{babel}
\usepackage[UKenglish]{isodate}

\usepackage{graphicx}
\usepackage{amsmath,amssymb}
\usepackage[font=small]{caption}

% TABLES
\usepackage{tabu}
\usepackage{longtable}
\usepackage{booktabs}

% PDF SETUP
\usepackage{hyperref}
\hypersetup
{
  pdfauthor={Kenrick Turner, Rachael Morris, Nicholas Levy},
  pdfsubject={},
  pdftitle={Perioperative and discharge analgesic practice for lower limb arthroplasty in the East of England},
  pdfkeywords={},
  unicode,
  hidelinks,
  breaklinks,
  xetex,
  bookmarks
}


% FONTS
\usepackage{fontspec}
\setmainfont{Palatino Linotype}
\setromanfont{Palatino Linotype}
\setsansfont{Myriad Pro}
\setmonofont[Scale=0.8]{Source Code Pro} %Scale=0.7 or Scale=MatchLowercase
\usepackage{authblk}


\title{Perioperative and discharge analgesic practice for lower limb arthroplasty in the East of England}
\author[$\dag$]{Kenrick Turner (CT2)}
\author[$\dag$]{Rachael Morris (ST6)}
\author[$\dag$]{Nicholas Levy (Consultant)}
\affil[$\dag$]{Department of Anaesthesia, West Suffolk Hospital NHS Foundation Trust}
\date{}



\begin{document}

\maketitle

\section*{Background}
Careless opioid prescribing has resulted in an epidemic with substantial medical and social consequences. Far from being an isolated problem in the US, it is increasingly prevalent in other healthcare systems including Australasia and the UK. From the US, several risk factors have been identified for inappropriate opioid prescribing: type of surgery; use of oxycodone; use of modified release (MR) preparations; large doses of discharge medication; lack of deprescribing advice.\cite{Shah:2017kn}

There are now concerns that prescribing practices of UK anaesthetists may predispose UK patients to subsequent opioid dependence.\cite{Levy:vWXr4C3m}

\section*{Aim}

To analyse current prescribing practice for perioperative analgesia in major lower limb arthroplasty in the region.

\section*{Methods}
Using lower limb arthroplasty as a risk factor for opioid dependence, we sought to collate the perioperative protocols for lower limb arthroplasty from the 14 hospitals in the region.

\section*{Results}
Nine of the fourteen hospitals responded. Eight had formal perioperative analgesic protocols.

All employed a multi-modal postoperative strategy that included MR opioids. All but one recognised the need to de-escalate, specifying a limit to the regular MR opioids.

Four made recommendations for discharge medication. One recommended a regular MR Morphine. None provided Oxycodone on discharge. Three discharged with Oramorph PRN.


\section*{Conclusion}
Encouragingly, most hospitals not only have a protocolised approach to perioperative analgesia, but also incorporate de-prescribing recommendations for modified-release opioids. Most hospitals are discharging with Oramorph rather than MR preparations, presenting a lower risk of opioid dependence.

However, hospitals continue to use MR opioids in their inpatient postoperative phase, which presents challenges in titration and weaning. This is of particular relevance given the recent ANZA position statement and the increasing recognition of the risks that MR preparations present in precipitating iatrogenic opioid dependence.\cite{Anonymous:2018is}

% BIBLIOGRAPHY
\bibliographystyle{naturemag}

\bibliography{EoEpostopanalgesia}

\end{document}
